%==============================================================================
\section{Experimental Evaluation}
\label{sec:experiments}
%==============================================================================

All experiments are conducted on Rust implementations of each algorithm on synthetic datasets of user objects (name, age, country), and executed on an Apple M3 Pro. We use Rust as the primary evaluation language due to its predictable performance characteristics, absence of garbage collection and JIT compilation, and low runtime variance. Execution times are measured after sufficient warm-up iterations to account for caching and allocator effects. Each reported data point corresponds to the mean over repeated runs, with a 95\% confidence interval of at most $\approx 5\%$. The benchmarking routines are provided in the repository for reproducibility~\cite{deltasort-repo}. We used fully random update distributions, which provide no favorable structure to any particular algorithm. We evaluated following algorithms:

\begin{itemize}
  \item \textbf{NativeSort}: Rust's \texttt{sort\_by} implementation based on PDQSort~\cite{peters2021pdqsort}, representing full re-sorting of the array. NativeSort serves as a critical baseline to identify the crossover threshold for each update-aware algorithm beyond which full re-sorting becomes preferable.

  \item \textbf{Binary-Insertion-Sort (BIS)} and \textbf{Extract--Sort--Merge (ESM)}: standard update-aware sorting algorithms that represent distinct time--space trade-offs, as described in \secref{sec:introduction}.

  \item \textbf{DeltaSort}: our proposed update-aware sorting algorithm, as described in \secref{sec:algorithm}.
\end{itemize}

\begin{table}[t]
\centering
\caption{Algorithm complexity comparison}
\label{tab:complexity}
\begin{tabular}{l c c c c}
\toprule
Algorithm & Comparisons (C) & Movement (M) & Space (S) & Time (C + M) \\
\midrule
FullSort$^{*}$ $(\texttt{sort\_by})$ & $O(n \log n)$ & - & - & $O(n \log n)$ \\
BIS & $O(k \log n)$ & $O(kn)$ & $O(1)$ & $O(kn)$ \\
ESM & $O(k \log k + n)$ & $O(n)$ & $O(n)$ & $O(k \log k + n)$ \\
\textbf{DeltaSort} & $O(k \log n)$ & $O(n\sqrt{k})$ & $O(k)$ & $O(k \log n + n \sqrt k)$$^{\dagger}$ \\
\bottomrule
\end{tabular}

\vspace{0.3em}
{\small $^{*}$Not update-aware; hence complexity is a function of only $n$. The exact movement and space complexity are implementation-dependent and not required for the present discussion.}


{\small $^{\dagger}$Expected. Worst-case time is $O(kn)$}
\end{table}

\subsection{Correctness}
Correctness is formally proven in \thmref{thm:correctness} and also verified by an extensive set of randomized tests~\cite{deltasort-repo} across various scales and update sizes. The test routine generates a sorted base array of size $n$, applies $k$ random updates at random indices, runs DeltaSort, and asserts that the final array is sorted and contains all original values with updated values.

\subsection{Execution Time}

\figref{fig:rust-execution-time} shows execution time (in \textmu s) for $n = 100$K values as a function of the number of updated values $k$ on a log--log scale. We use a log--log scale to highlight interesting behavior at lower ranges of $k$, which is the most practically relevant regime. As $k$ increases, all update-aware algorithms eventually lose to NativeSort (the \emph{crossover threshold}), as the overhead of processing updates begins to dominate.

% Figure: All algorithms comparison (log-log scale)
\begin{figure}[H]
\centering
\begin{tikzpicture}
\begin{axis}[
    width=0.8\textwidth,
    height=5cm,
    xlabel={Number of updated values ($k$)},
    ylabel={Execution time (\textmu s)},
    xmode=log,
    ymode=log,
    log basis x=10,
    log basis y=10,
    xmin=1, xmax=100000,
    ymin=5, ymax=100000,
    xtick={1, 10, 100, 1000, 10000, 100000},
    xticklabels={1, 10 (0.01\%), 100 (0.1\%), 1K (1\%), 10K (10\%), 100K},
    legend pos=south east,
    legend style={font=\small, fill=white, fill opacity=0.95, draw=gray!50},
    grid=major,
    major grid style={line width=0.3pt, draw=gray!30},
    tick label style={font=\small},
    label style={font=\small},
]

% FullSort
\addplot[color=gray!70, mark=square*, thick, mark size=2pt] 
    table[col sep=comma, x=k, y=native] {\rootdir/figures/js/execution-time.csv};

% Binary Insertion
\addplot[color=orange!80, mark=triangle*, thick, mark size=2pt] 
    table[col sep=comma, x=k, y=bis] {\rootdir/figures/js/execution-time.csv};
% Extract-Sort-Merge
\addplot[color=purple!70, mark=diamond*, thick, mark size=2pt] 
    table[col sep=comma, x=k, y=esm] {\rootdir/figures/js/execution-time.csv};

% DeltaSort (emphasized)
\addplot[color=green!70!black, mark=*, thick, mark size=2.5pt, line width=1.5pt] 
    table[col sep=comma, x=k, y=deltasort] {\rootdir/figures/js/execution-time.csv};

\legend{FullSort, BIS, ESM, \textbf{DeltaSort}}
\end{axis}
\end{tikzpicture}
\caption{Execution time comparison for JavaScript implementations for $n = 100$K (log-log scale).}
\label{fig:js-execution-time}
\end{figure}


Several observations emerge from the execution-time profile:

\begin{enumerate}
  \item \textbf{Asymptotic trend is consistent}: For larger values of $k$, the observed execution times for all algorithms closely follow their expected asymptotic trends. In particular, BIS exhibits superlinear growth consistent with its $O(kn)$ movement cost, while ESM and NativeSort converge toward $O(n \log n)$ behavior, respectively.

  \item \textbf{Binary-Insertion-Sort degrades rapidly}: BIS outperforms ESM and NativeSort for very small update sizes ($k \lesssim 0.01\%$) but quickly becomes the worst-performing algorithm as $k$ increases due to its $O(kn)$ data movement cost.

  \item \textbf{Extract--Sort--Merge dominates the mid-range}: ESM is the fastest baseline across most of the intermediate range ($1\% \lesssim k \lesssim 80\%$), provided sufficient memory is available. Its performance comes at the cost of $O(n)$ auxiliary space.

  \item \textbf{DeltaSort dominates a narrow but meaningful regime}: DeltaSort outperforms both BIS and ESM for $k \lesssim 1\%$. It uses $O(k)$ auxiliary space—more than BIS ($O(1)$) but substantially less than ESM ($O(n)$). While this regime is small in absolute terms, it aligns well with practical workloads where deltas are usually a small percentage of the full dataset.
\end{enumerate}

These observations suggest that, much like hybrid blind sorting algorithms (e.g., TimSort~\cite{timsort}, PDQSort~\cite{peters2021pdqsort}), it would be beneficial to construct \emph{hybrid update-aware} strategies that adapt to update size and execution environment. For the Rust implementation evaluated here, an optimal strategy for an environment with no memory constraints could be:

\begin{center}
  Use DeltaSort for $k \lesssim 1\%$, ESM for $1\% \lesssim k \lesssim 80\%$, and NativeSort for $k \lesssim 80\%$.
\end{center}

The precise crossover thresholds depend on the execution environment, data distribution, and comparator cost, and should be treated as indicative rather than universal. We will explore these hybrid strategies further in future work.

\subsection{Comparison Count}

\figref{fig:rust-comparator-count} shows the total number of comparisons as a function of update count $k$. We have used a realistic multi-key comparator that compares user objects by (country, age, name) to reflect practical comparison costs.

% Figure: Comparator invocation count comparison
\begin{figure}[H]
\centering
\begin{tikzpicture}
\begin{axis}[
    width=0.92\textwidth,
    height=6cm,
    xlabel={Number of updated values ($k$)},
    ylabel={Comparator invocations},
    xmode=log,
    ymode=log,
    log basis x=10,
    log basis y=10,
    xmin=1, xmax=100000,
    ymin=10, ymax=10000000,
    xtick={1, 10, 100, 1000, 10000},
    xticklabels={1, 10, 100, 1K, 10K},
    ytick={10, 100, 1000, 10000, 100000, 1000000, 10000000},
    yticklabels={10, 100, 1K, 10K, 100K, 1M, 10M},
    legend pos=south east,
    legend style={font=\small},
    grid=both,
    grid style={line width=0.1pt, draw=gray!30},
    major grid style={line width=0.2pt, draw=gray!50},
]

% Native Sort
\addplot[color=gray!70, mark=square*, thick, mark size=2.5pt] 
    table[col sep=comma, x=k, y=native] {\rootdir/figures/rust/comparator-count.csv};

% Binary Insertion
\addplot[color=orange!80, mark=triangle*, thick, mark size=2.5pt] 
    table[col sep=comma, x=k, y=bis] {\rootdir/figures/rust/comparator-count.csv};

% Extract-Sort-Merge
\addplot[color=purple!70, mark=diamond*, thick, mark size=2.5pt] 
    table[col sep=comma, x=k, y=esm] {\rootdir/figures/rust/comparator-count.csv};

% DeltaSort (emphasized)
\addplot[color=green!70!black, mark=*, thick, mark size=2.5pt, line width=1.5pt] 
    table[col sep=comma, x=k, y=deltasort] {\rootdir/figures/rust/comparator-count.csv};

\legend{NativeSort, BIS, ESM, \textbf{DeltaSort}}
\end{axis}
\end{tikzpicture}
\caption{Comparator invocation count for $n = 50$K. DeltaSort and BIS
both achieve $O(k \log n)$ comparisons, while NativeSort uses $O(n \log n)$ regardless
of $k$, and ESM uses $O(k \log k + n)$. The 10--40\% higher comparison counts for DeltaSort
vs BIS confirm that movement confinement (Lemma~\ref{lem:confinement}) is the major factor in DeltaSort speedup.\rustbenchmarknote}
\label{fig:rust-comparator-count}
\end{figure}


Several clear patterns emerge:

\begin{enumerate}
  \item \textbf{Asymptotic behavior matches theory}: The observed comparator counts closely follow the  expected asymptotic trends for all algorithms across the full range of $k$.

  \item \textbf{ESM incurs high comparison cost even for small updates}: Due to the full merge step, ESM performs a large number of comparisons even when $k$ is small.

  \item \textbf{DeltaSort matches BIS asymptotically}: DeltaSort's comparator count closely tracks BIS for $k \lesssim 1\%$ and remains within $\approx 10$--$20\%$ of BIS for larger $k$, while preserving the same $O(k \log n)$ asymptotic behavior. Note that for large $k$, even though BIS is most comparison-efficient, its data movement cost makes it an impractical choice.
\end{enumerate}

The key takeaway is that DeltaSort, like BIS, substantially reduces redundant comparison work relative to ESM and NativeSort. As comparator cost increases, this advantage would becomes more pronounced. For small update ranges, where BIS and DeltaSort have similar comparison counts, DeltaSort's movement efficiency becomes the dominant factor, and its relative advantage over BIS will persist across a wide range of comparator costs.

\subsection{Crossover Threshold}

Every update-aware algorithm would eventually be beaten by NativeSort as $k$ increases, once the overhead of processing the updates outweighs its benefits. For each algorithm, we calculate the threshold $k_c$ via binary search across the range of $k$. \figref{fig:rust-crossover-all} shows the crossover thresholds for all three update-aware algorithms versus NativeSort.

% Figure: Crossover ratio vs array size (all algorithms vs native)
\begin{figure}[H]
\centering
\begin{tikzpicture}
\begin{axis}[
    width=0.92\textwidth,
    height=7cm,
    xlabel={Array size ($n$)},
    ylabel={Crossover ratio $k_c / n$ (\%)},
    xmode=log,
    log basis x=10,
    xmin=1000, xmax=1000000,
    ymin=0, ymax=100,
    ytick={0, 20, 40, 60, 80, 100},
    xtick={1000,2000, 5000, 10000, 20000, 50000, 100000,200000, 500000, 1000000},
    xticklabels={1K, 2K, 5K, 10K, 20K, 50K, 100K, 200K, 500K, 1M},
    legend pos=north west,
    legend style={font=\small, fill=white, fill opacity=0.95, draw=gray!50},
    grid=both,
    grid style={line width=0.1pt, draw=gray!20},
    major grid style={line width=0.2pt, draw=gray!40},
    tick label style={font=\small},
    label style={font=\small},
]

% BIS crossover (O(1) space, but slow O(kn) time)
\addplot[color=orange!80, mark=triangle*, thick, mark size=2.5pt, line width=1.2pt] 
    table[col sep=comma, x=n, y=bis] {\rootdir/figures/rust/crossover-all.csv};

% ESM crossover (O(n) space, fast O(n + k log k) time)
\addplot[color=purple!70, mark=diamond*, thick, mark size=2.5pt, line width=1.2pt] 
    table[col sep=comma, x=n, y=esm] {\rootdir/figures/rust/crossover-all.csv};

% DeltaSort crossover (O(k) space, balanced performance)
\addplot[color=green!70!black, mark=*, thick, mark size=2.5pt, line width=1.5pt] 
    table[col sep=comma, x=n, y=deltasort] {\rootdir/figures/rust/crossover-all.csv};

% Space annotations
\node[font=\scriptsize, text=orange!70!black, anchor=west] at (axis cs:12000000, 8) {$O(1)$ space};
\node[font=\scriptsize, text=green!60!black, anchor=west] at (axis cs:12000000, 35) {$O(k)$ space};
\node[font=\scriptsize, text=purple!60!black, anchor=west] at (axis cs:12000000, 75) {$O(n)$ space};

\legend{BIS, ESM, \textbf{DeltaSort}}
\end{axis}
\end{tikzpicture}
\caption{Crossover thresholds for update-aware algorithms versus NativeSort. Each line shows the maximum update fraction $k_c/n$ at which the algorithm outperforms full re-sorting. BIS uses $O(1)$ space but has a low crossover due to $O(kn)$ time complexity. ESM achieves the highest crossover with $O(n)$ space. DeltaSort occupies the middle ground with $O(k)$ space.\rustbenchmarknote}
\label{fig:rust-crossover-all}
\end{figure}

Several patterns emerge:

\begin{enumerate}
  \item \textbf{BIS has a very low crossover threshold}: Due to its $O(kn)$ data movement cost, BIS is competitive only for very small update sizes.

  \item \textbf{ESM achieves the highest crossover threshold}: ESM remains faster than NativeSort up to $k \approx 60$--$80\%$, consistent with its $O(n + k \log k)$ time complexity, which approaches full sorting only when $k$ is large.

  \item \textbf{DeltaSort exhibits an intermediate threshold}: DeltaSort's crossover point lies between BIS and ESM, aligning with its $O(k \log n)$ time complexity, which interpolates between the two extremes.

  \item \textbf{Thresholds are weakly dependent on array size}:
  Crossover points are largely stable across array sizes, though we observe a decline for large arrays ($n \lesssim 200$K). This may be due to cache effects or memory allocation overheads. We will study this behavior in more detail in future work.
\end{enumerate}

The key takeaway is that DeltaSort occupies a well-defined middle ground among update-aware algorithms. It supports substantially larger update batches than BIS while requiring significantly less auxiliary space than ESM, making it attractive in regimes where both time and space constraints matter.


An interesting trend in \figref{fig:rust-execution-time} is that for $k \lesssim 1\%$, DeltaSort consistently outperforms ESM while using significantly less auxiliary space, making it a \emph{strictly better choice} in this regime. To examine this boundary more closely, we plot the direct crossover between DeltaSort and ESM in \figref{fig:rust-crossover-ds-vs-esm}.

% Figure: DeltaSort vs ESM crossover threshold
\begin{figure}[H]
\centering
\begin{tikzpicture}
\begin{axis}[
    width=0.92\textwidth,
    height=6.5cm,
    xlabel={Array size ($n$)},
    ylabel={Crossover ratio $k_c / n$ (\%)},
    xmode=log,
    log basis x=10,
    xmin=500, xmax=20000000,
    ymin=0, ymax=100,
    ytick={0, 20, 40, 60, 80, 100},
    xtick={1000, 10000, 100000, 1000000, 10000000},
    xticklabels={1K, 10K, 100K, 1M, 10M},
    legend pos=north east,
    legend style={font=\small, fill=white, fill opacity=0.95, draw=gray!50},
    grid=both,
    grid style={line width=0.1pt, draw=gray!20},
    major grid style={line width=0.2pt, draw=gray!40},
    tick label style={font=\small},
    label style={font=\small},
]

% Shaded region below the curve (DeltaSort wins)
\addplot[name path=curve, color=green!70!black, mark=*, thick, mark size=2.5pt, line width=1.5pt] 
    table[col sep=comma, x=n, y=crossover_ratio] {\rootdir/figures/rust/crossover-ds-vs-esm.csv};
\path[name path=bottom] (axis cs:1000,0) -- (axis cs:10000000,0);
\addplot[green!15, opacity=0.6] fill between[of=curve and bottom];

% Annotation for the regions
\node[font=\footnotesize, text=green!50!black] at (axis cs:50000, 15) {DeltaSort faster};
\node[font=\footnotesize, text=purple!50!black] at (axis cs:50000, 85) {ESM faster};

\legend{\textbf{DeltaSort} vs ESM}
\end{axis}
\end{tikzpicture}
\caption{Crossover threshold where DeltaSort's performance advantage over ESM diminishes. Below the curve (shaded region), DeltaSort is faster while using only $O(k)$ space compared to ESM's $O(n)$ space. This demonstrates that DeltaSort provides both time and space benefits for moderate update sizes.\rustbenchmarknote}
\label{fig:rust-crossover-ds-vs-esm}
\end{figure}


\figref{fig:rust-crossover-ds-vs-esm} shows that there exists a non-trivial regime in which DeltaSort dominates ESM both in execution time and space usage. For example, for an array of size $n = 100$K, DeltaSort outperforms ESM for up to $\approx 1K$ updates, while using only $\approx 1\%$ of the auxiliary space required by ESM. As array size increases, this regime steadily shrinks. For large arrays, the crossover threshold moves toward smaller fractions, indicating that DeltaSort loses to ESM earlier. Although both algorithms incur $O(n)$ data movement in the worst case, this trend suggests that the constant factors associated with DeltaSort's movement become more pronounced at larger scales. As a result, ESM's simpler linear merge begins to dominate for smaller $k$ as $n$ grows.

\subsection{Performance in managed execution environments}

DeltaSort was also implemented in JavaScript~\cite{deltasort-repo} and benchmarked on Node v22 to evaluate behavior in managed runtimes to study how factors such as garbade collections, JIT compilation and runtime optimizations impact performance. \figref{fig:js-execution-time} shows execution time for $n = 100$K as a function of update size $k$.

% Figure: All algorithms comparison (log-log scale)
\begin{figure}[H]
\centering
\begin{tikzpicture}
\begin{axis}[
    width=0.8\textwidth,
    height=5cm,
    xlabel={Number of updated values ($k$)},
    ylabel={Execution time (\textmu s)},
    xmode=log,
    ymode=log,
    log basis x=10,
    log basis y=10,
    xmin=1, xmax=100000,
    ymin=5, ymax=100000,
    xtick={1, 10, 100, 1000, 10000, 100000},
    xticklabels={1, 10 (0.01\%), 100 (0.1\%), 1K (1\%), 10K (10\%), 100K},
    legend pos=south east,
    legend style={font=\small, fill=white, fill opacity=0.95, draw=gray!50},
    grid=major,
    major grid style={line width=0.3pt, draw=gray!30},
    tick label style={font=\small},
    label style={font=\small},
]

% FullSort
\addplot[color=gray!70, mark=square*, thick, mark size=2pt] 
    table[col sep=comma, x=k, y=native] {\rootdir/figures/js/execution-time.csv};

% Binary Insertion
\addplot[color=orange!80, mark=triangle*, thick, mark size=2pt] 
    table[col sep=comma, x=k, y=bis] {\rootdir/figures/js/execution-time.csv};
% Extract-Sort-Merge
\addplot[color=purple!70, mark=diamond*, thick, mark size=2pt] 
    table[col sep=comma, x=k, y=esm] {\rootdir/figures/js/execution-time.csv};

% DeltaSort (emphasized)
\addplot[color=green!70!black, mark=*, thick, mark size=2.5pt, line width=1.5pt] 
    table[col sep=comma, x=k, y=deltasort] {\rootdir/figures/js/execution-time.csv};

\legend{FullSort, BIS, ESM, \textbf{DeltaSort}}
\end{axis}
\end{tikzpicture}
\caption{Execution time comparison for JavaScript implementations for $n = 100$K (log-log scale).}
\label{fig:js-execution-time}
\end{figure}


Several observations stand out:

\begin{enumerate}
  \item \textbf{DeltaSort and BIS largely overlap for small updates}: For $k \lesssim 0.1\%$, even though BIS and DeltaSort beat ESM and NativeSort, DeltaSort does not exhibit a clear advantage over Binary-Insertion-Sort. Unlike in Rust, batching updates does not help improve performance meaningfully in this regime.

  \item \textbf{ESM beats NativeSort}: Extract--Sort--Merge performs better than NativeSort for $k$ up to $\approx 50\%$.

  \item \textbf{DeltaSort's advantage diminishes in managed runtimes}:
  JavaScript runtimes like V8 do not guarantee contiguous array layout~\cite{v8elementskinds}, and movement costs are not proportional to the number of shifted elements. As a result, DeltaSort's core optimization---reducing physical data movement through segmentation---does not translate into consistent wall-clock improvements.
\end{enumerate}

The key takeaway is that \emph{DeltaSort's performance benefits rely on a predictable movement cost model}, which holds in low-level, unmanaged environments (e.g., Rust) but not in managed runtimes such as V8. In such environments, the practical strategy simplifies to:

\begin{center}
Use Binary-Insertion-Sort for $k \llless n$. Use Extract--Sort--Merge or NativeSort for $k \approx O(n)$.
\end{center}

These results highlight that update-aware sorting algorithms must be evaluated together with the execution semantics of the target runtime, rather than assuming uniform cost models across environments. \appref{sec:appendix-js} provides additional data about crossover thesholds in JS.