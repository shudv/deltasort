%==============================================================================
\section{Future Work}
\label{sec:future}
%==============================================================================

This work opens up several directions for further investigation.

First, while our analysis focuses on a bounded-range update model with uniformly random updates, many practical workloads exhibit additional structure. Examples include bounded-window updates, value perturbation models, temporally correlated or clustered updates. These models may improve or regress the expected movement and comparison costs of DeltaSort, potentially impacting the regime in which it is applicable. Hence, a more systematic study of such update models is necessary.

Second, we have seen how segments can be fixed locally and independently. This strongly suggests opportunities for parallel execution, where different segments could be fixed concurrently without interference. In contrast, Binary-Insertion-Sort is inherently sequential due to overlapping in-place shifts, and Extract–Sort–Merge parallelizes only partially while incurring substantial auxiliary space costs. Exploring parallel variants of DeltaSort and understanding their scalability on multi-core systems is a promising direction for future work.

Finally, while this paper treats DeltaSort as a standalone algorithm for analysis, practical systems will benefit from adaptive hybrid strategies that select among Binary-Insertion-Sort, DeltaSort, Extract–Sort–Merge, and full re-sorting based on observed update sizes and system constraints. Designing heuristic-based, low-overhead mechanisms for such dynamic selection remains an open challenge.
