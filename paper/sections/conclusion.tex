%==============================================================================
\section{Conclusion}
\label{sec:conclusion}
%==============================================================================

In this paper, we studied the problem of maintaining sorted arrays under incremental \emph{known} updates. We argued that this \emph{update-aware} setting arises naturally in many practical systems. Within this model, we presented \emph{DeltaSort}, an incremental sorting algorithm that batches multiple updates instead of applying them independently. The key idea is to exploit structure created by pre-sorting updated values, which induces directional segmentation and enables localized, non-overlapping fixes. This enables DeltaSort to reduce redundant data movement. Our theoretical analysis shows that DeltaSort matches the comparison efficiency of BIS while exhibiting $O(n \sqrt k)$ data movement under the random bounded-range update model. Experimental results in Rust demonstrate that DeltaSort occupies a distinct space in the incremental sorting design spectrum: it outperforms BIS and is more space-efficient than ESM across a wider range of $k$. To stress-test the underlying cost assumptions, we also evaluated DeltaSort in the V8 runtime, where memory movement is less predictable. In this setting, the advantages observed in Rust largely disappear, highlighting that DeltaSort’s benefits depend on execution environments with transparent and contiguous memory layouts.

Overall, this work shows that exposing update information to lower-level sorting primitives enables new algorithmic trade-offs that are fundamentally unavailable to blind sorting algorithms.