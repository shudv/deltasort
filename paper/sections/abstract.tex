% Abstract
Sorting values or records is a fundamental operation in many systems. When records need to be read in a particular order, sorting at read time incurs repeated $O(n \log n)$ cost and can become a bottleneck in read-heavy workloads. A common solution is to maintain a derived sorted read-replica that is kept updated asynchronously as the underlying data changes. For updating read-replicas that are stored as arrays, existing approaches rely either on full re-sorting or on incremental techniques such as repeated binary insertion, which incurs high data movement, or merge-based strategies, which require linear auxiliary space. In this paper, we study incremental sorting under a model in which the sorting routine is explicitly informed of the indices of elements updated since the previous sort—a setting that naturally arises in systems that track update deltas. Under this model, we present \emph{DeltaSort}, a new algorithm for incremental sorting that occupies an intermediate point in the time-space trade-off spectrum. We provide theoretical analysis and experimental evidence showing that, for random update distributions, \emph{DeltaSort} achieves lower execution time than insertion-based incremental sorting while using substantially less auxiliary space than merge-based approaches.