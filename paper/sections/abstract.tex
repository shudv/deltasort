% Abstract
Sorting values or records is a fundamental operation in many systems. When records need to be read in a particular order, sorting-on-read incurs repeated $O(n \log n)$ cost which can become a bottleneck in read-heavy systems. A standard solution is to have a derived sorted read-replica that is updated asynchronously whenever the system-of-record gets updated. For updating read-replicas stored as arrays, existing approaches rely on either full re-sorting or incremental techniques such as Binary-Insertion-Sort or Extract–Sort–Merge, each providing different time-space tradeoffs. In this paper, we study incremental sorting under an alternative model where the sorting routine is explicitly informed of the indices of updated values since the previous sort. Under this model, we present \emph{DeltaSort}, a new algorithm that fills a previously empty region in the time-space tradeoff spectrum. We present theoretical and experimental analysis showing that, for random update distributions, \emph{DeltaSort} is multi-fold faster than Binary-Insertion-Sort ($O(k \log n + n \sqrt k)$ vs. $O(kn)$) while using only $O(k)$ auxiliary space—making it the first algorithm that outperforms full re-sorting for non-trivial update batches without requiring $O(n)$ extra space.