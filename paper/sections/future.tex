%==============================================================================
\section{Future Work}
\label{sec:future}
%==============================================================================

While our analysis focuses on a random bounded-range update model, many practical workloads exhibit additional structure. Examples include \emph{bounded-rank} or \emph{clustered} updates. These models may improve or worsen the movement and comparison costs of DeltaSort, potentially impacting the range in which it is applicable. Hence, a more systematic study of such update models is important.

Additionally, we showed how segments can be fixed \emph{locally} and \emph{independently}. This strongly suggests opportunities for \emph{parallel execution}, where different segments could be fixed concurrently without interference. In contrast, BIS is inherently sequential due to overlapping in-place shifts, and even though ESM parallelizes quite well, it uses substantial space. Exploring the parallel variant of DeltaSort and understanding its scalability on multi-core systems is a promising direction for future work.

We also saw that DeltaSort's performance advantage relies on predictable memory movement costs, which does not hold in runtimes like V8. Further study is needed to understand whether this is a general limitation or specific to V8.

Finally, while this paper treats DeltaSort as a standalone algorithm for analysis, practical systems would benefit from \emph{hybrid adaptive strategies} that select among BIS, DeltaSort, ESM, and full re-sorting based on update sizes and system constraints. Designing heuristic-based, low-overhead mechanisms for such dynamic selection is an open challenge.
