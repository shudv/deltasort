\appendix

\section{Comparison Count}
\label{sec:appendix-comparisons}

\figref{fig:rust-comparator-count} shows the total number of comparisons as a function of update count $k$.

% Figure: Comparator invocation count comparison
\begin{figure}[H]
\centering
\begin{tikzpicture}
\begin{axis}[
    width=0.92\textwidth,
    height=6cm,
    xlabel={Number of updated values ($k$)},
    ylabel={Comparator invocations},
    xmode=log,
    ymode=log,
    log basis x=10,
    log basis y=10,
    xmin=1, xmax=100000,
    ymin=10, ymax=10000000,
    xtick={1, 10, 100, 1000, 10000},
    xticklabels={1, 10, 100, 1K, 10K},
    ytick={10, 100, 1000, 10000, 100000, 1000000, 10000000},
    yticklabels={10, 100, 1K, 10K, 100K, 1M, 10M},
    legend pos=south east,
    legend style={font=\small},
    grid=both,
    grid style={line width=0.1pt, draw=gray!30},
    major grid style={line width=0.2pt, draw=gray!50},
]

% Native Sort
\addplot[color=gray!70, mark=square*, thick, mark size=2.5pt] 
    table[col sep=comma, x=k, y=native] {\rootdir/figures/rust/comparator-count.csv};

% Binary Insertion
\addplot[color=orange!80, mark=triangle*, thick, mark size=2.5pt] 
    table[col sep=comma, x=k, y=bis] {\rootdir/figures/rust/comparator-count.csv};

% Extract-Sort-Merge
\addplot[color=purple!70, mark=diamond*, thick, mark size=2.5pt] 
    table[col sep=comma, x=k, y=esm] {\rootdir/figures/rust/comparator-count.csv};

% DeltaSort (emphasized)
\addplot[color=green!70!black, mark=*, thick, mark size=2.5pt, line width=1.5pt] 
    table[col sep=comma, x=k, y=deltasort] {\rootdir/figures/rust/comparator-count.csv};

\legend{NativeSort, BIS, ESM, \textbf{DeltaSort}}
\end{axis}
\end{tikzpicture}
\caption{Comparator invocation count for $n = 50$K. DeltaSort and BIS
both achieve $O(k \log n)$ comparisons, while NativeSort uses $O(n \log n)$ regardless
of $k$, and ESM uses $O(k \log k + n)$. The 10--40\% higher comparison counts for DeltaSort
vs BIS confirm that movement confinement (Lemma~\ref{lem:confinement}) is the major factor in DeltaSort speedup.\rustbenchmarknote}
\label{fig:rust-comparator-count}
\end{figure}


\section{DeltaSort vs ESM Crossover Threshold}
\label{sec:appendix-rust-ds-vs-esm}

% Figure: DeltaSort vs ESM crossover threshold
\begin{figure}[H]
\centering
\begin{tikzpicture}
\begin{axis}[
    width=0.92\textwidth,
    height=6.5cm,
    xlabel={Array size ($n$)},
    ylabel={Crossover ratio $k_c / n$ (\%)},
    xmode=log,
    log basis x=10,
    xmin=500, xmax=20000000,
    ymin=0, ymax=100,
    ytick={0, 20, 40, 60, 80, 100},
    xtick={1000, 10000, 100000, 1000000, 10000000},
    xticklabels={1K, 10K, 100K, 1M, 10M},
    legend pos=north east,
    legend style={font=\small, fill=white, fill opacity=0.95, draw=gray!50},
    grid=both,
    grid style={line width=0.1pt, draw=gray!20},
    major grid style={line width=0.2pt, draw=gray!40},
    tick label style={font=\small},
    label style={font=\small},
]

% Shaded region below the curve (DeltaSort wins)
\addplot[name path=curve, color=green!70!black, mark=*, thick, mark size=2.5pt, line width=1.5pt] 
    table[col sep=comma, x=n, y=crossover_ratio] {\rootdir/figures/rust/crossover-ds-vs-esm.csv};
\path[name path=bottom] (axis cs:1000,0) -- (axis cs:10000000,0);
\addplot[green!15, opacity=0.6] fill between[of=curve and bottom];

% Annotation for the regions
\node[font=\footnotesize, text=green!50!black] at (axis cs:50000, 15) {DeltaSort faster};
\node[font=\footnotesize, text=purple!50!black] at (axis cs:50000, 85) {ESM faster};

\legend{\textbf{DeltaSort} vs ESM}
\end{axis}
\end{tikzpicture}
\caption{Crossover threshold where DeltaSort's performance advantage over ESM diminishes. Below the curve (shaded region), DeltaSort is faster while using only $O(k)$ space compared to ESM's $O(n)$ space. This demonstrates that DeltaSort provides both time and space benefits for moderate update sizes.\rustbenchmarknote}
\label{fig:rust-crossover-ds-vs-esm}
\end{figure}


\figref{fig:rust-crossover-ds-vs-esm} shows that there exists a non-trivial range in which DeltaSort dominates ESM both in execution time and space usage. For example, for an array of size $n = 100$K, DeltaSort outperforms ESM for up to $\approx 1K$ updates, while using only $\approx 1\%$ of the auxiliary space required by ESM. As array size increases, this range steadily shrinks. For large arrays, the crossover threshold moves toward smaller fractions, indicating that DeltaSort loses to ESM earlier. More study is needed to understand this trend fully.

\section{Update-aware algorithms evaluated in JavaScript/V8}
\label{sec:appendix-js-deltasort}

% Figure: All algorithms comparison (log-log scale)
\begin{figure}[H]
\centering
\begin{tikzpicture}
\begin{axis}[
    width=0.8\textwidth,
    height=5cm,
    xlabel={Number of updated values ($k$)},
    ylabel={Execution time (\textmu s)},
    xmode=log,
    ymode=log,
    log basis x=10,
    log basis y=10,
    xmin=1, xmax=100000,
    ymin=5, ymax=100000,
    xtick={1, 10, 100, 1000, 10000, 100000},
    xticklabels={1, 10 (0.01\%), 100 (0.1\%), 1K (1\%), 10K (10\%), 100K},
    legend pos=south east,
    legend style={font=\small, fill=white, fill opacity=0.95, draw=gray!50},
    grid=major,
    major grid style={line width=0.3pt, draw=gray!30},
    tick label style={font=\small},
    label style={font=\small},
]

% FullSort
\addplot[color=gray!70, mark=square*, thick, mark size=2pt] 
    table[col sep=comma, x=k, y=native] {\rootdir/figures/js/execution-time.csv};

% Binary Insertion
\addplot[color=orange!80, mark=triangle*, thick, mark size=2pt] 
    table[col sep=comma, x=k, y=bis] {\rootdir/figures/js/execution-time.csv};
% Extract-Sort-Merge
\addplot[color=purple!70, mark=diamond*, thick, mark size=2pt] 
    table[col sep=comma, x=k, y=esm] {\rootdir/figures/js/execution-time.csv};

% DeltaSort (emphasized)
\addplot[color=green!70!black, mark=*, thick, mark size=2.5pt, line width=1.5pt] 
    table[col sep=comma, x=k, y=deltasort] {\rootdir/figures/js/execution-time.csv};

\legend{FullSort, BIS, ESM, \textbf{DeltaSort}}
\end{axis}
\end{tikzpicture}
\caption{Execution time comparison for JavaScript implementations for $n = 100$K (log-log scale).}
\label{fig:js-execution-time}
\end{figure}


\begin{enumerate}
  \item DeltaSort and BIS show almost identical performance profiles. This indicates that \emph{the underlying memory movement cost model in V8 is not proportional to the number of shifted values}. As a result, DeltaSort's core optimization---reducing physical data movement through segmentation---does not translate into wall-clock improvements. This could be because V8 has different kinds of array representations~\cite{v8elementskinds} that may not guarantee contiguous layouts.
  \item ESM is $\sim 40\%$ faster than FullSort (\texttt{Array.prototype.sort}) for $k$ up to $\approx 50\%$. This indicates that \textbf{even in managed runtimes, we can leverage information of updated indices to unlock better performance}.
\end{enumerate}

The key takeaway is that \emph{DeltaSort's performance benefits rely on a predictable movement cost model}, which holds in low-level, unmanaged environments (e.g., Rust) but not in managed runtimes such as V8. Hence in V8, the practical strategy for achieving faster execution times simplifies to:

\begin{center}
Use BIS for $k \ll n$, ESM for $k \lesssim 40\%$, \texttt{Array.prototype.sort} for $k \gtrsim 40\%$.
\end{center}

These results highlight that update-aware sorting algorithms must be evaluated together with the execution semantics of the target runtime, rather than assuming uniform cost models across environments. \appref{sec:appendix-js} provides additional data about crossover thresholds in JS.

\section{Crossover Thresholds in JavaScript}
\label{sec:appendix-js}

\figref{fig:js-crossover-all} and~\figref{fig:js-crossover-ds-vs-esm} show crossover thresholds for the JavaScript implementations. The thresholds are generally lower than in Rust due to the managed runtime's memory movement behavior. BIS and DeltaSort have almost identical behavior in JavaScript, while ESM maintains a higher threshold ($\approx 35$--$40\%$). This implies that for small $k$, BIS is preferable to DeltaSort in JavaScript because of lower space requirement and simpler implementation, unlike in Rust where DeltaSort has a clear measurable advantage.

% Figure: Crossover ratio vs array size (all algorithms vs native)
\begin{figure}[H]
\centering
\begin{tikzpicture}
\begin{axis}[
    width=0.92\textwidth,
    height=7cm,
    xlabel={Array size ($n$)},
    ylabel={Crossover ratio $k_c / n$ (\%)},
    xmode=log,
    log basis x=10,
    xmin=1000, xmax=1000000,
    ymin=0, ymax=100,
    ytick={0, 20, 40, 60, 80, 100},
    xtick={1000,2000, 5000, 10000, 20000, 50000, 100000,200000, 500000, 1000000},
    xticklabels={1K, 2K, 5K, 10K, 20K, 50K, 100K, 200K, 500K, 1M},
    legend pos=north west,
    legend style={font=\small, fill=white, fill opacity=0.95, draw=gray!50},
    grid=both,
    grid style={line width=0.1pt, draw=gray!20},
    major grid style={line width=0.2pt, draw=gray!40},
    tick label style={font=\small},
    label style={font=\small},
]

% BIS crossover (O(1) space, but slow O(kn) time)
\addplot[color=orange!80, mark=triangle*, thick, mark size=2.5pt, line width=1.2pt] 
    table[col sep=comma, x=n, y=bis] {\rootdir/figures/rust/crossover-all.csv};

% ESM crossover (O(n) space, fast O(n + k log k) time)
\addplot[color=purple!70, mark=diamond*, thick, mark size=2.5pt, line width=1.2pt] 
    table[col sep=comma, x=n, y=esm] {\rootdir/figures/rust/crossover-all.csv};

% DeltaSort crossover (O(k) space, balanced performance)
\addplot[color=green!70!black, mark=*, thick, mark size=2.5pt, line width=1.5pt] 
    table[col sep=comma, x=n, y=deltasort] {\rootdir/figures/rust/crossover-all.csv};

% Space annotations
\node[font=\scriptsize, text=orange!70!black, anchor=west] at (axis cs:12000000, 8) {$O(1)$ space};
\node[font=\scriptsize, text=green!60!black, anchor=west] at (axis cs:12000000, 35) {$O(k)$ space};
\node[font=\scriptsize, text=purple!60!black, anchor=west] at (axis cs:12000000, 75) {$O(n)$ space};

\legend{BIS, ESM, \textbf{DeltaSort}}
\end{axis}
\end{tikzpicture}
\caption{Crossover thresholds for update-aware algorithms versus NativeSort. Each line shows the maximum update fraction $k_c/n$ at which the algorithm outperforms full re-sorting. BIS uses $O(1)$ space but has a low crossover due to $O(kn)$ time complexity. ESM achieves the highest crossover with $O(n)$ space. DeltaSort occupies the middle ground with $O(k)$ space.\rustbenchmarknote}
\label{fig:rust-crossover-all}
\end{figure}


% Figure: DeltaSort vs ESM crossover threshold
\begin{figure}[H]
\centering
\begin{tikzpicture}
\begin{axis}[
    width=0.92\textwidth,
    height=6.5cm,
    xlabel={Array size ($n$)},
    ylabel={Crossover ratio $k_c / n$ (\%)},
    xmode=log,
    log basis x=10,
    xmin=500, xmax=20000000,
    ymin=0, ymax=100,
    ytick={0, 20, 40, 60, 80, 100},
    xtick={1000, 10000, 100000, 1000000, 10000000},
    xticklabels={1K, 10K, 100K, 1M, 10M},
    legend pos=north east,
    legend style={font=\small, fill=white, fill opacity=0.95, draw=gray!50},
    grid=both,
    grid style={line width=0.1pt, draw=gray!20},
    major grid style={line width=0.2pt, draw=gray!40},
    tick label style={font=\small},
    label style={font=\small},
]

% Shaded region below the curve (DeltaSort wins)
\addplot[name path=curve, color=green!70!black, mark=*, thick, mark size=2.5pt, line width=1.5pt] 
    table[col sep=comma, x=n, y=crossover_ratio] {\rootdir/figures/rust/crossover-ds-vs-esm.csv};
\path[name path=bottom] (axis cs:1000,0) -- (axis cs:10000000,0);
\addplot[green!15, opacity=0.6] fill between[of=curve and bottom];

% Annotation for the regions
\node[font=\footnotesize, text=green!50!black] at (axis cs:50000, 15) {DeltaSort faster};
\node[font=\footnotesize, text=purple!50!black] at (axis cs:50000, 85) {ESM faster};

\legend{\textbf{DeltaSort} vs ESM}
\end{axis}
\end{tikzpicture}
\caption{Crossover threshold where DeltaSort's performance advantage over ESM diminishes. Below the curve (shaded region), DeltaSort is faster while using only $O(k)$ space compared to ESM's $O(n)$ space. This demonstrates that DeltaSort provides both time and space benefits for moderate update sizes.\rustbenchmarknote}
\label{fig:rust-crossover-ds-vs-esm}
\end{figure}

