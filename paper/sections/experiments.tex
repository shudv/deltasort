%==============================================================================
\section{Experimental Evaluation}
\label{sec:experiments}
%==============================================================================

All experiments \footnote{Along with experiments, we also validated correctness by an extensive set of randomized tests~\cite{deltasort-repo} across various scales of $n$ and full range of $k$.} are conducted on Rust implementations of each algorithm on synthetic datasets of user objects (name, age, country), and executed on an Apple M3 Pro. We use Rust as the primary evaluation language due to its predictable performance characteristics \footnote{In contrast, managed execution environments like JavaScript on V8 have various factors like garbage collection, non-contiguous memory layouts~\cite{v8elementskinds} etc. that impact the underlying cost model.}. Execution times are measured after sufficient warm-up iterations to account for caching and allocator effects. Each reported data point corresponds to the mean over repeated runs, with a 95\% confidence interval of at most $5\%$. The benchmarking routines are provided in the repository for reproducibility~\cite{deltasort-repo}. We used fully random update distributions, which provide no favorable structure to any particular algorithm. We evaluated following algorithms:

\begin{itemize}
  \item \textbf{NativeSort}: Rust's \texttt{sort\_by} implementation based on PDQSort~\cite{peters2021pdqsort}, representing full re-sorting of the array. NativeSort serves as a critical baseline to identify the crossover threshold $k_c$ for each update-aware algorithm beyond which full re-sorting becomes preferable.

  \item \textbf{Binary-Insertion-Sort (BIS)} and \textbf{Extract--Sort--Merge (ESM)}: standard update-aware sorting algorithms that represent distinct time--space trade-offs, as described in \secref{sec:introduction}.

  \item \textbf{DeltaSort}: our proposed update-aware sorting algorithm, as described in \secref{sec:algorithm}.
\end{itemize}

\begin{table}[t]
\centering
\caption{Algorithm complexity comparison}
\label{tab:complexity}
\begin{tabular}{l c c c c}
\toprule
Algorithm & Comparisons (C) & Movement (M) & Space (S) & Time (C + M) \\
\midrule
FullSort$^{*}$ $(\texttt{sort\_by})$ & $O(n \log n)$ & - & - & $O(n \log n)$ \\
BIS & $O(k \log n)$ & $O(kn)$ & $O(1)$ & $O(kn)$ \\
ESM & $O(k \log k + n)$ & $O(n)$ & $O(n)$ & $O(k \log k + n)$ \\
\textbf{DeltaSort} & $O(k \log n)$ & $O(n\sqrt{k})$ & $O(k)$ & $O(k \log n + n \sqrt k)$$^{\dagger}$ \\
\bottomrule
\end{tabular}

\vspace{0.3em}
{\small $^{*}$Not update-aware; hence complexity is a function of only $n$. The exact movement and space complexity are implementation-dependent and not required for the present discussion.}


{\small $^{\dagger}$Expected. Worst-case time is $O(kn)$}
\end{table}

\subsection{Execution Time}

\figref{fig:rust-execution-time} shows execution time (in \textmu s) for $n = 100$K values as a function of the number of updated values $k$ on a log--log scale. We use a log--log scale to highlight interesting behavior at lower ranges of $k$, which is the most practically relevant regime. As $k$ increases, all update-aware algorithms eventually lose to NativeSort (\emph{crossover threshold}), as the overhead of processing updates begins to overshadow any benefit from harnessing the presortedness.

% Figure: All algorithms comparison (log-log scale)
\begin{figure}[H]
\centering
\begin{tikzpicture}
\begin{axis}[
    width=0.8\textwidth,
    height=5cm,
    xlabel={Number of updated values ($k$)},
    ylabel={Execution time (\textmu s)},
    xmode=log,
    ymode=log,
    log basis x=10,
    log basis y=10,
    xmin=1, xmax=100000,
    ymin=5, ymax=100000,
    xtick={1, 10, 100, 1000, 10000, 100000},
    xticklabels={1, 10 (0.01\%), 100 (0.1\%), 1K (1\%), 10K (10\%), 100K},
    legend pos=south east,
    legend style={font=\small, fill=white, fill opacity=0.95, draw=gray!50},
    grid=major,
    major grid style={line width=0.3pt, draw=gray!30},
    tick label style={font=\small},
    label style={font=\small},
]

% FullSort
\addplot[color=gray!70, mark=square*, thick, mark size=2pt] 
    table[col sep=comma, x=k, y=native] {\rootdir/figures/js/execution-time.csv};

% Binary Insertion
\addplot[color=orange!80, mark=triangle*, thick, mark size=2pt] 
    table[col sep=comma, x=k, y=bis] {\rootdir/figures/js/execution-time.csv};
% Extract-Sort-Merge
\addplot[color=purple!70, mark=diamond*, thick, mark size=2pt] 
    table[col sep=comma, x=k, y=esm] {\rootdir/figures/js/execution-time.csv};

% DeltaSort (emphasized)
\addplot[color=green!70!black, mark=*, thick, mark size=2.5pt, line width=1.5pt] 
    table[col sep=comma, x=k, y=deltasort] {\rootdir/figures/js/execution-time.csv};

\legend{FullSort, BIS, ESM, \textbf{DeltaSort}}
\end{axis}
\end{tikzpicture}
\caption{Execution time comparison for JavaScript implementations for $n = 100$K (log-log scale).}
\label{fig:js-execution-time}
\end{figure}


Several observations emerge from the execution-time profile:

\begin{enumerate}
  \item The asymptotic behavior for each algorithm aligns with theoretical expectations. BIS exhibits superlinear growth consistent with its $O(kn)$ movement cost, while ESM and DeltaSort converge toward $O(n \log n)$ behavior as $k$ approaches $n$.
  \item \textbf{DeltaSort is the fastest baseline for} $k \lesssim 1\%$. It uses more auxiliary space than BIS ($O(k)$ vs $O(1)$) but substantially less than ESM ($O(k)$ vs $O(n)$). This regime is small in absolute terms, but aligns well with practical workloads where deltas are usually a small percentage of the full dataset.

  \item ESM is the fastest baseline across most of the intermediate range ($1\% \lesssim k \lesssim 80\%$). Although, its performance comes at the cost of $O(n)$ auxiliary space. DeltaSort provides a more space-efficient alternative to ESM for $k \lesssim 30\%$, albeit with higher execution time.
\end{enumerate}

These observations suggest that, much like hybrid blind sorting algorithms (e.g., TimSort~\cite{timsort}, PDQSort~\cite{peters2021pdqsort}), it would be beneficial to construct \emph{hybrid update-aware} strategies that are adaptive. As an example, for the Rust implementation evaluated here, a balanced strategy for an environment that requires fast execution without excessive space usage would be:

\begin{center}
  Use DeltaSort for $k \lesssim 10\%$, ESM for $10\% \lesssim k \lesssim 80\%$, and NativeSort for $k \gtrsim 80\%$.
\end{center}

The optimal crossover thresholds for a scenario would depend on several factors like memory availability, distribution of update sizes etc. The key takeaway is that DeltaSort occupies a distinct and practically relevant point in the time--space trade-off spectrum for update-aware sorting.

\subsection{Comparison Count}
Comparator cost is another factor that influences performance and varies widely across scenarios. For our evaluation, we picked a multi-key comparator that is a realistic proxy for practical use cases. Our experiments show that DeltaSort’s comparator invocation count closely tracks that of Binary-Insertion-Sort, consistent with the $O(k \log n)$ theoretical bound. As a result, when comparator cost dominates, DeltaSort retains its advantage over Binary-Insertion-Sort while further widening its gap relative to Extract–Sort–Merge, whose comparison overhead is higher. This shifts the crossover thresholds in favor of DeltaSort and expands the regimes where it is the preferred choice. Detailed comparison counts for all algorithms are provided in Appendix~\ref{sec:appendix-comparisons}.

\subsection{Crossover Threshold}

Every update-aware algorithm would eventually be beaten by NativeSort as $k$ increases, once the overhead of processing the updates outweighs its benefits. For each algorithm, we calculate the threshold $k_c$ via binary search across the range of $k$. \figref{fig:rust-crossover-all} shows the crossover thresholds for all three update-aware algorithms versus NativeSort.

% Figure: Crossover ratio vs array size (all algorithms vs native)
\begin{figure}[H]
\centering
\begin{tikzpicture}
\begin{axis}[
    width=0.92\textwidth,
    height=7cm,
    xlabel={Array size ($n$)},
    ylabel={Crossover ratio $k_c / n$ (\%)},
    xmode=log,
    log basis x=10,
    xmin=1000, xmax=1000000,
    ymin=0, ymax=100,
    ytick={0, 20, 40, 60, 80, 100},
    xtick={1000,2000, 5000, 10000, 20000, 50000, 100000,200000, 500000, 1000000},
    xticklabels={1K, 2K, 5K, 10K, 20K, 50K, 100K, 200K, 500K, 1M},
    legend pos=north west,
    legend style={font=\small, fill=white, fill opacity=0.95, draw=gray!50},
    grid=both,
    grid style={line width=0.1pt, draw=gray!20},
    major grid style={line width=0.2pt, draw=gray!40},
    tick label style={font=\small},
    label style={font=\small},
]

% BIS crossover (O(1) space, but slow O(kn) time)
\addplot[color=orange!80, mark=triangle*, thick, mark size=2.5pt, line width=1.2pt] 
    table[col sep=comma, x=n, y=bis] {\rootdir/figures/rust/crossover-all.csv};

% ESM crossover (O(n) space, fast O(n + k log k) time)
\addplot[color=purple!70, mark=diamond*, thick, mark size=2.5pt, line width=1.2pt] 
    table[col sep=comma, x=n, y=esm] {\rootdir/figures/rust/crossover-all.csv};

% DeltaSort crossover (O(k) space, balanced performance)
\addplot[color=green!70!black, mark=*, thick, mark size=2.5pt, line width=1.5pt] 
    table[col sep=comma, x=n, y=deltasort] {\rootdir/figures/rust/crossover-all.csv};

% Space annotations
\node[font=\scriptsize, text=orange!70!black, anchor=west] at (axis cs:12000000, 8) {$O(1)$ space};
\node[font=\scriptsize, text=green!60!black, anchor=west] at (axis cs:12000000, 35) {$O(k)$ space};
\node[font=\scriptsize, text=purple!60!black, anchor=west] at (axis cs:12000000, 75) {$O(n)$ space};

\legend{BIS, ESM, \textbf{DeltaSort}}
\end{axis}
\end{tikzpicture}
\caption{Crossover thresholds for update-aware algorithms versus NativeSort. Each line shows the maximum update fraction $k_c/n$ at which the algorithm outperforms full re-sorting. BIS uses $O(1)$ space but has a low crossover due to $O(kn)$ time complexity. ESM achieves the highest crossover with $O(n)$ space. DeltaSort occupies the middle ground with $O(k)$ space.\rustbenchmarknote}
\label{fig:rust-crossover-all}
\end{figure}

Several patterns emerge:

\begin{enumerate}
  \item \textbf{BIS has a very low crossover threshold}: Due to its $O(kn)$ data movement cost, BIS is competitive only for very small update sizes.

  \item \textbf{ESM achieves the highest crossover threshold}: ESM remains faster than NativeSort up to $k \approx 60$--$80\%$, consistent with its $O(n + k \log k)$ time complexity, which approaches full sorting only when $k$ is large.

  \item \textbf{DeltaSort exhibits an intermediate threshold}: DeltaSort's crossover point lies between BIS and ESM, aligning with its $O(k \log n)$ time complexity, which interpolates between the two extremes.
  
  \item \textbf{DeltaSort vs ESM}: For $k \lesssim 1\%$, DeltaSort outperforms ESM while using significantly less auxiliary space, making it a \emph{strictly better choice} in this regime. Appendix~\ref{sec:appendix-rust-ds-vs-esm} shows how this boundary shifts with array size.

  \item \textbf{Thresholds are weakly dependent on array size}:
  Crossover points are largely stable across array sizes, though we observe a decline for large arrays ($n \lesssim 200$K). This may be due to cache effects or memory allocation overheads. We will study this behavior in more detail in future work.
\end{enumerate}

The key takeaway is that DeltaSort occupies a well-defined middle ground among update-aware algorithms. It supports substantially larger update batches than BIS while requiring significantly less auxiliary space than ESM, making it attractive in regimes where both time and space constraints matter.

\subsection{Performance in managed execution environments}

DeltaSort was also implemented in JavaScript~\cite{deltasort-repo} and benchmarked on Node v22 to evaluate behavior in managed runtimes to study how factors such as garbade collections, JIT compilation and runtime optimizations impact performance. \figref{fig:js-execution-time} shows execution time for $n = 100$K as a function of update size $k$.

% Figure: All algorithms comparison (log-log scale)
\begin{figure}[H]
\centering
\begin{tikzpicture}
\begin{axis}[
    width=0.8\textwidth,
    height=5cm,
    xlabel={Number of updated values ($k$)},
    ylabel={Execution time (\textmu s)},
    xmode=log,
    ymode=log,
    log basis x=10,
    log basis y=10,
    xmin=1, xmax=100000,
    ymin=5, ymax=100000,
    xtick={1, 10, 100, 1000, 10000, 100000},
    xticklabels={1, 10 (0.01\%), 100 (0.1\%), 1K (1\%), 10K (10\%), 100K},
    legend pos=south east,
    legend style={font=\small, fill=white, fill opacity=0.95, draw=gray!50},
    grid=major,
    major grid style={line width=0.3pt, draw=gray!30},
    tick label style={font=\small},
    label style={font=\small},
]

% FullSort
\addplot[color=gray!70, mark=square*, thick, mark size=2pt] 
    table[col sep=comma, x=k, y=native] {\rootdir/figures/js/execution-time.csv};

% Binary Insertion
\addplot[color=orange!80, mark=triangle*, thick, mark size=2pt] 
    table[col sep=comma, x=k, y=bis] {\rootdir/figures/js/execution-time.csv};
% Extract-Sort-Merge
\addplot[color=purple!70, mark=diamond*, thick, mark size=2pt] 
    table[col sep=comma, x=k, y=esm] {\rootdir/figures/js/execution-time.csv};

% DeltaSort (emphasized)
\addplot[color=green!70!black, mark=*, thick, mark size=2.5pt, line width=1.5pt] 
    table[col sep=comma, x=k, y=deltasort] {\rootdir/figures/js/execution-time.csv};

\legend{FullSort, BIS, ESM, \textbf{DeltaSort}}
\end{axis}
\end{tikzpicture}
\caption{Execution time comparison for JavaScript implementations for $n = 100$K (log-log scale).}
\label{fig:js-execution-time}
\end{figure}


Several observations stand out:

\begin{enumerate}
  \item \textbf{DeltaSort and BIS largely overlap for small updates}: For $k \lesssim 0.1\%$, even though BIS and DeltaSort beat ESM and NativeSort, DeltaSort does not exhibit a clear advantage over Binary-Insertion-Sort. Unlike in Rust, batching updates does not help improve performance meaningfully in this regime.

  \item \textbf{ESM beats NativeSort}: Extract--Sort--Merge performs better than NativeSort for $k$ up to $\approx 50\%$.

  \item \textbf{DeltaSort's advantage diminishes in managed runtimes}:
  JavaScript runtimes like V8 do not guarantee contiguous array layout~\cite{v8elementskinds}, and movement costs are not proportional to the number of shifted elements. As a result, DeltaSort's core optimization---reducing physical data movement through segmentation---does not translate into consistent wall-clock improvements.
\end{enumerate}

The key takeaway is that \emph{DeltaSort's performance benefits rely on a predictable movement cost model}, which holds in low-level, unmanaged environments (e.g., Rust) but not in managed runtimes such as V8. In such environments, the practical strategy simplifies to:

\begin{center}
Use Binary-Insertion-Sort for $k \llless n$. Use Extract--Sort--Merge or NativeSort for $k \approx O(n)$.
\end{center}

These results highlight that update-aware sorting algorithms must be evaluated together with the execution semantics of the target runtime, rather than assuming uniform cost models across environments. \appref{sec:appendix-js} provides additional data about crossover thesholds in JS.