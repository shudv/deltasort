%==============================================================================
\section{Conclusion}
\label{sec:conclusion}
%==============================================================================

This paper studied the problem of maintaining sorted arrays under incremental \emph{known} updates. We argued that this \emph{update-aware} setting arises naturally in many practical systems, yet is not directly addressed by standard blind sorting routines. Within this model, we presented \emph{DeltaSort}, an incremental sorting algorithm that batches multiple updates instead of applying them independently. The key idea is to exploit structure created by pre-sorting updated values, which induces directional segmentation and enables localized, non-overlapping fixes. This allows DeltaSort to reduce redundant data movement.

Our theoretical analysis shows that DeltaSort matches the comparison efficiency of Binary-Insertion-Sort while providing stronger guarantees on data movement under reasonable update models. Experimental results in Rust demonstrate that DeltaSort occupies a distinct space in the incremental sorting design spectrum: it outperforms Binary-Insertion-Sort for small update batches, is more space-efficient than Extract–Sort–Merge across a wider range of updates, and exhibits clear crossover points beyond which other strategies become preferable. Importantly, the results also highlight the limits of this approach. DeltaSort does not dominate existing techniques across all kinds of execution environments, and its advantages depend on predictable memory movement costs. In managed runtimes and environments with less transparent memory layouts, these benefits may diminish.

Overall, this work shows that exposing update information to the sorting routine enables new algorithmic trade-offs that are not available to blind sorting algorithms. DeltaSort demonstrates that even within the well-studied domain of sorting, modest changes to the problem formulation can yield practically useful algorithmic techniques for incremental workloads.
